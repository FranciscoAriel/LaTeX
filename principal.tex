% !TeX root = Principal.tex
% !TeX encoding = ISO-8859-1
\documentclass[oneside,12pt]{book}                              %%Voy a imprimir las hojas por un solo lado, letras tamaño 12
\usepackage[hang,small,bf]{caption}                             %%Títulos de las figuras
\usepackage{subfig}                                             %%Poner juntas dos o más figuras
\usepackage{fancyhdr}                                           %%Encabezados perrones
\usepackage{amssymb}                                            %%Símbolos de la AMS
\usepackage[latin1]{inputenc}                                   %%Poner acentos, ñ, etc.
\usepackage{amsfonts}                                           %%Fuentes de la AMS
\usepackage{amsmath}
\usepackage{amsthm}
\usepackage[spanish,activeacute]{babel}
\usepackage{lscape}                                              %%Para poder poner paginas volteadas
\usepackage[sort,round,comma]{natbib}                            %%Para citas
\usepackage{longtable}                                           %%Para Tablas largas
\usepackage{multirow}                                            %%Para unir filas en Tablas
\usepackage{geometry}                                            %%Geometría de la página
\usepackage{pdfpages}                                            %%Insertar documentos completos en pdf, solo funciona con pdflatex
\usepackage{setspace}                                            %%Espacios entre figuras, Tablas y sus leyendas
\usepackage{float}                                               %%Permite forzar el lugar donde aparecen las Tablas, figuras y hacer otros trucos
\usepackage{listings}                                            %%Listados de programas en C++, C, R, etc....
\usepackage{hyperref}                                            %%Ligas en citas, referencias a fórmulas, url, etc.
\hypersetup{pdftitle=Título de la tesis,
            pdfauthor=Autor,
            pdfsubject=Tesis de maestría,
            colorlinks,
            citecolor=blue,
            filecolor=blue,
            linkcolor=blue,
            urlcolor=blue} %Necesario para facilitar lectura
%%\hypersetup{colorlinks,citecolor=black,filecolor=black,linkcolor=black,urlcolor=black} %Necesario para facilitar lectura

\geometry{letterpaper,tmargin=1 in,bmargin=1 in,lmargin=1.25 in,rmargin=1.00 in}%
\singlespacing                                                   %%Espaciado simple %%
%%\onehalfspacing                                                 %%Espacio y medio %%

\setlength{\captionmargin}{20pt}                                 %%Figuras
\setlength{\abovecaptionskip}{10pt}
\setlength{\belowcaptionskip}{10pt}
\vfuzz2pt

\providecommand{\references}{}
\renewcommand{\references}{\small \bibliographystyle{mybibsty}}  %%Un estilo de citas en español
\parindent 0cm                                                   %%Párrafos sin sangrias
\parskip  0.5cm                                                  %%Salto entre párrafos
\decimalpoint                                                    %%punto como separador decimal

%%%%%%%%%%%%%%%%%%%%%%%%%%%%%%%%%%%%%%%%%%%%
%%Esto es para poner Encabezados
%%%%%%%%%%%%%%%%%%%%%%%%%%%%%%%%%%%%%%%%%%%%
\pagestyle{fancy}
\renewcommand{\chaptermark}[1]{\markright{\thechapter.\ #1}}
\renewcommand{\sectionmark}[1]{\markright{\thesection.\ #1}{}}
%%\fancyhead[LO,LE]{\bfseries\rightmark}
%%\fancyhead[RO,RE]{}
\fancyhead[L]{\nouppercase{\bfseries\rightmark}}
\fancyhead[R]{}
\cfoot{\thepage}
\headheight 16pt

%%%%%%%%%%%%%%%%%%%%%%%%%%%%%%%%%%%%%%%%%%%
%%Esto es para NO poner Encabezados
%%%%%%%%%%%%%%%%%%%%%%%%%%%%%%%%%%%%%%%%%%%
%%\pagestyle{plain}
%%%%%%%%%%%%%%%%%%%%%%%%%%%%%%%%%%%%%%%%%%%

%%%%%%%%%%%%%%%%%%%%%%%%%%%%%%%%%%%%%%%%%%%
%%Silabeo
%%\hyphenation{ge-ne-ra-li-za-da na-tu-ra-le-za de-no-mi-na-dor ve-ro-si-mi-li-tu-des ve-ro-si-mi-li-tud}
%%%%%%%%%%%%%%%%%%%%%%%%%%%%%%%%%%%%%%%%%%%

% %\newtheorem{resultado}{Resultado}[chapter]
%%\newenvironment{prueba}[1][Prueba]{\textbf{#1} \newline}
%\newenvironment{prueba}[1][Prueba]{\noindent\textbf{#1} \newline }{\ \rule{0.5em}{0.5em}}
\newtheorem{defi}{Definición}[chapter]
\newtheorem{teo}{Teorema}[chapter]

%%Alias para citas
%\defcitealias{Rsoft}{R}                                        %%R es un alias para Rsoft que se encuentra en referecias.bib
                                                               %%Al momento de hacer las citas aparecerá R cuando se llame \citetalias{Rsoft}
%\defcitealias{sn_package}{\bf{sn}}                             %%sn es una alias a sn_package

%\lstset{language=C}
%-------------Sin que rompa palabras (interlineado)-------
\tolerance=1
\emergencystretch=\maxdimen
\hyphenpenalty=10000
\hbadness=10000
\begin{document}

%%%%%%%%%%%%%%%%%%%%%%%%%%%%%%%%%%%%%%%%%%%%%%%%%%%%%%%%%%%%%
%% Páginas iniciales de la tesis:
%% Aquí se definen el título de la tesis, el autor de la misma,
%% así como el título del resumen en inglés.
%% El logo de la institución se obtiene de una carpeta llamada
%% figuras en el directorio donde se encuentra ESTE archivo.
%%%%%%%%%%%%%%%%%%%%%%%%%%%%%%%%%%%%%%%%%%%%%%%%%%%%%%%%%%%%%%

\newcommand{\titulo}[1]{\def\eltitulo{#1}}
\newcommand{\tituloen}[1]{\def\eltituloen{#1}}
\newcommand{\autor}[1]{\def\elautor{#1}}

\titulo{Título}
\tituloen{Title}
\autor{Autor}

\thispagestyle{empty}
\pagenumbering{roman}

%% Barra izquierda - Escudos
\hskip-1.5cm
\begin{minipage}[c][1\totalheight][s]{3cm}
  \begin{center}
    \includegraphics[height=2.5cm]{figuras/logo_cp.pdf}\\[10pt]
    \hskip2pt\vrule width2pt height19.5cm\hskip1mm
    \vrule width1pt height19.5cm\\[10pt]
  \end{center}
\end{minipage}\quad
%% Barra derecha - Títulos
\begin{minipage}[c][1\totalheight][s]{13.5cm}
  \begin{center}
    % Barra superior
    {\huge COLEGIO DE POSTGRADUADOS}
    \vspace{.3cm}
    \hrule height2pt
    \vspace{.1cm}
    \hrule height1pt
    \vspace{.8cm}
    {\large INSTITUCIÓN DE ENSEÑANZA E INVESTIGACIÓN\\EN CIENCIAS AGRÍCOLAS}


    \vspace{1.5cm}
    {\large \textbf{CAMPUS MONTECILLO}}

    \vspace{1.5cm}
    {SOCIOECONOMÍA, ESTADÍSTICA E INFORMÁTICA\\ESTADÍSTICA}

    \vspace{1.5cm}
    {\large \textbf{\eltitulo}}

    \vspace{1.5cm}
    {\large \elautor}

    \vspace{1.5cm}
    {\Large T E S I S}

    \vspace{1.5cm}
    {\large PRESENTADA COMO REQUISITO PARCIAL}
    {\large PARA OBTENER EL GRADO DE:}

    \vspace{1.5cm}
    {\Large MAESTRO EN CIENCIAS}

    \vspace{1.71cm}
    {\large{MONTECILLO,TEXCOCO, EDO. DE MÉXICO\\2015}}
    \vspace{0.5cm}

    {\hskip-2.4cm
       \begin{minipage}[c][1\totalheight][s]{16.5cm}
         \hrule width15cm height1pt\vskip1mm
         \hrule width15cm height2pt
       \end{minipage}
    }
  \end{center}
\end{minipage}

\endinput 

\newpage
\thispagestyle{plain}
La presente tesis titulada: \textbf{\eltitulo},
realizada por el alumno: \textbf{\elautor}, bajo la dirección
del Consejo Particular indicado ha sido aprobada por el mismo y aceptada
como requisito parcial para obtener el grado de:

\begin{center}
    \large \textbf{MAESTRO EN CIENCIAS} \\
    \vspace{1.5cm}
    \large \textbf{SOCIOECONOMÍA, ESTADÍSTICA E INFORMÁTICA\\ESTADÍSTICA}\\
    \vspace{1.5cm}
    \large \textbf{CONSEJO PARTICULAR}
    \vspace{1.5cm}

    \begin{tabular}{ll}
    {\textbf CONSEJERO} & \rule{7cm}{0.1mm} \\
    { } & { Dr. consejero} \\[1.2cm]
    {\textbf ASESOR} & \rule{7cm}{0.1mm} \\
    {} & { Dr. asesor 1} \\[1.2cm]
    {\textbf ASESOR} & \rule{7cm}{0.1mm} \\
    { } & { Dr. asesor 2} \\[1.2cm]
    {\textbf ASESOR} & \rule{7cm}{0.1mm} \\
    { } & { Dr. externo} \\[1.2cm]
    \end{tabular}
\end{center}

\endinput 

\newpage
\thispagestyle{plain}
\begin{center}
  {\large \textbf{\eltitulo}}\\
  \vskip 0.5cm
  \elautor \\
  \vskip 0.5cm
  Colegio de Postgraduados, 2005\\
\end{center}

\vskip 0.5cm

\noindent En el presente trabajo ...

\vskip 0.5cm
\noindent \textbf{Palabras clave:} Regresión Poisson, Algoritmo EM, Exceso de ceros.


\endinput  

\newpage
\thispagestyle{plain}
\begin{center}
  {\large \textbf{\eltituloen}}\\
  \vskip 0.5cm
  \elautor\\
  \vskip 0.5cm
  Colegio de Postgraduados, 2005\\
\end{center}

\vskip 0.5cm

\noindent In this work ...

\vskip 0.5cm
\noindent \textbf{Key words:} Poisson regression, EM algorithm, Zero-inflated.

\endinput  

\newpage
\thispagestyle{plain}

\begin{center}
  \large{\textbf{AGRADECIMIENTOS}}
\end{center}

Al Consejo Nacional de Ciencia y Tecnología (CONACYT) por el apoyo
económico brindado durante la realización de mis estudios.

Al Colegio de Postgraduados, por haberme brindado la oportunidad de seguir
mi formación académica en sus aulas.

A los integrantes de mi Consejo Particular:

Al Dr.  mi más sincero agradecimiento, por sus
atinadas indicaciones y consejos mismos que me fueron de gran utilidad en la
realización del presente trabajo de tesis.

A mis asesores Dr.  y Dr.  por su orientación, apoyo y
colaboración desinteresada en el presente trabajo.

Al Dr. , por sus consejos tan atinados y por dedicar parte de
su tiempo en la revisión de este trabajo de tesis.

A mis profesores, compañeros de clase y todas aquellas personas que de
alguna manera me apoyaron en esta tarea, a todos gracias.

\endinput  

\newpage
\thispagestyle{plain}

\begin{center}
  \large{\textbf{DEDICATORIA}}
\end{center}

A mis padres  y mis hermanos Isabel{\ldots} y

\vskip 0.5cm

A alguien especial.

\endinput 


\renewcommand{\labelitemi}{$\bullet$}                            %%Cambia la etiqueta para el primer nivel de itemize, despliega un circulo sólido
                                                                 %%Para acceder a otros niveles, usar \labelitemii, \labelitemiii, \labelitemiv
\renewcommand{\contentsname}{Índice}                             %%Cambia Índice general por Índice de contenido
\renewcommand{\tablename}{Tabla}                                 %%Cambia la palabra Cuadro que aparece por default por Tabla
\renewcommand{\listtablename}{Índice de tablas}                  %%Cambia Índice de Cuadros por Índice de Figuras
\renewcommand{\bibname}{Referencias}                             %%Cambia bibliografía por Referencias

\tableofcontents
\listoftables
\listoffigures

\mainmatter
\pagenumbering{arabic}
\setcounter{page}{1}
%%%%%%%%%%%%%%%%%%%%%%%%%%%%%%%%%%%%%%%%%%%%%%%%%%%%%%%%%%%%%%%%%%%%%%%%%%%%%%%%%%%%%%%%%%%%%%%%%%%%%%%%%%%%%%%%%
\chapter{Introducción}
%%%%%%%%%%%%%%%%%%%%%%%%%%%%%%%%%%%%%%%%%%%%%%%%%%%%%%%%%%%%%%%%%%%%%%%%%%%%%%%%%%%%%%%%%%%%%%%%%%%%%%%%%%%%%%%%%
\thispagestyle{empty}

%%%%%%%%%%%%%%%%%%%%%%%%%%%%%%%%%%%%%%%%%%%%%%%%%%%%%%%%%%%%%%%%%%%%%%%%%%%%%%%%%%%%%%%%%%%%%%%%%%%%%%%%%%%%%%%%%
%%Referencias
%%%%%%%%%%%%%%%%%%%%%%%%%%%%%%%%%%%%%%%%%%%%%%%%%%%%%%%%%%%%%%%%%%%%%%%%%%%%%%%%%%%%%%%%%%%%%%%%%%%%%%%%%%%%%%%%%
\addcontentsline{toc}{chapter}{\:\,\quad Referencias}
\references
\bibliography{biblio}
%%%%%%%%%%%%%%%%%%%%%%%%%%%%%%%%%%%%%%%%%%%%%%%%%%%%%%%%%%%%%%%%%%%%%%%%%%%%%%%%%%%%%%%%%%%%%%%%%%%%%%%%%%%%%%%%%
\appendix
\chapter{Códigos en \texttt{R}}
\begin{verbatim}
for(i in 1:10)
{
  print(i)
}
\end{document}
